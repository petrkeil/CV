%%%%%%%%%%%%%%%%%%%%%%%%%%%%%%%%%%%%%%%%%%%%%%%%%%%%%%%%%%%%%%%%%%%%%%%%
%%%%%%%%%%%%%%%%%%%%%% Simple LaTeX CV Template %%%%%%%%%%%%%%%%%%%%%%%%
%%%%%%%%%%%%%%%%%%%%%%%%%%%%%%%%%%%%%%%%%%%%%%%%%%%%%%%%%%%%%%%%%%%%%%%%

%%%%%%%%%%%%%%%%%%%%%%%%%%%%%%%%%%%%%%%%%%%%%%%%%%%%%%%%%%%%%%%%%%%%%%%%
%% NOTE: If you find that it says                                     %%
%%                                                                    %%
%%                           1 of ??                                  %%
%%                                                                    %%
%% at the bottom of your first page, this means that the AUX file     %%
%% was not available when you ran LaTeX on this source. Simply RERUN  %%
%% LaTeX to get the ``??'' replaced with the number of the last page  %%
%% of the document. The AUX file will be generated on the first run   %%
%% of LaTeX and used on the second run to fill in all of the          %%
%% references.                                                        %%
%%%%%%%%%%%%%%%%%%%%%%%%%%%%%%%%%%%%%%%%%%%%%%%%%%%%%%%%%%%%%%%%%%%%%%%%

%%%%%%%%%%%%%%%%%%%%%%%%%%%% Document Setup %%%%%%%%%%%%%%%%%%%%%%%%%%%%

% Don't like 10pt? Try 11pt or 12pt
\documentclass[10pt]{article}

% The automated optical recognition software used to digitize resume
% information works best with fonts that do not have serifs. This
% command uses a sans serif font throughout. Uncomment both lines (or at
% least the second) to restore a Roman font (i.e., a font with serifs).
%\usepackage{times}
%\renewcommand{\familydefault}{\sfdefault}

% This is a helpful package that puts math inside length specifications
\usepackage{calc}

 % reverse order of lists
\usepackage{etaremune}
\usepackage{graphicx}
\usepackage{wrapfig}
\usepackage{multicol}

%\renewcommand{\familydefault}{\sfdefault}
%\usepackage[sfdefault]{FiraSans} %% option 'sfdefault' activates Fira Sans as the default text font
%\usepackage[T1]{fontenc}
%\renewcommand*\oldstylenums[1]{{\firaoldstyle #1}}

\usepackage[utf8]{inputenc}

% add thin horizontal lines
\newcommand{\HRule}{\rule{\linewidth}{0.05mm}}


% Layout: Puts the section titles on left side of page
\reversemarginpar

%
%         PAPER SIZE, PAGE NUMBER, AND DOCUMENT LAYOUT NOTES:
%
% The next \usepackage line changes the layout for CV style section
% headings as marginal notes. It also sets up the paper size as either
% letter or A4. By default, letter was used. If A4 paper is desired,
% comment out the letterpaper lines and uncomment the a4paper lines.
%
% As you can see, the margin widths and section title widths can be
% easily adjusted.
%
% ALSO: Notice that the includefoot option can be commented OUT in order
% to put the PAGE NUMBER *IN* the bottom margin. This will make the
% effective text area larger.
%
% IF YOU WISH TO REMOVE THE ``of LASTPAGE'' next to each page number,
% see the note about the +LP and -LP lines below. Comment out the +LP
% and uncomment the -LP.
%
% IF YOU WISH TO REMOVE PAGE NUMBERS, be sure that the includefoot line
% is uncommented and ALSO uncomment the \pagestyle{empty} a few lines
% below.
%

%% Use these lines for letter-sized paper
\usepackage[paper=letterpaper,
            %includefoot, % Uncomment to put page number above margin
            marginparwidth=1.2in,     % Length of section titles
            marginparsep=.05in,       % Space between titles and text
            margin=1in,               % 1 inch margins
            includemp]{geometry}

%% Use these lines for A4-sized paper
%\usepackage[paper=a4paper,
%            %includefoot, % Uncomment to put page number above margin
%            marginparwidth=30.5mm,    % Length of section titles
%            marginparsep=1.5mm,       % Space between titles and text
%            margin=25mm,              % 25mm margins
%            includemp]{geometry}

%% More layout: Get rid of indenting throughout entire document
\setlength{\parindent}{0in}

\usepackage[shortlabels]{enumitem}

% Simpler bibsections for CV sections
% (thanks to natbib for inspiration)
%
% * For lists of references with hanging indents and no numbers:
%
%   \begin{bibsection}
%       \item ...
%   \end{bibsection}
%
% * For numbered lists of references (with hanging indents):
%
%   \begin{bibenum}
%       \item ...
%   \end{bibenum}
%
%   Note that bibenum numbers continuously throughout. To reset the
%   counter, use
%
%   \restartlist{bibenum}
%
%   at the place where you want the numbering to reset.

\makeatletter
\newlength{\bibhang}
\setlength{\bibhang}{1em}
\newlength{\bibsep}
 {\@listi \global\bibsep\itemsep \global\advance\bibsep by\parsep}
\newlist{bibsection}{itemize}{3}
\setlist[bibsection]{label=,leftmargin=\bibhang,%
        itemindent=-\bibhang,
        itemsep=\bibsep,parsep=\z@,partopsep=0pt,
        topsep=0pt}
\newlist{bibenum}{enumerate}{3}
\setlist[bibenum]{label=[\arabic*],resume,leftmargin={\bibhang+\widthof{[999]}},%
        itemindent=-\bibhang,
        itemsep=\bibsep,parsep=\z@,partopsep=0pt,
        topsep=0pt}
\let\oldendbibenum\endbibenum
\def\endbibenum{\oldendbibenum\vspace{-.6\baselineskip}}
\let\oldendbibsection\endbibsection
\def\endbibsection{\oldendbibsection\vspace{-.6\baselineskip}}
\makeatother

%% Reference the last page in the page number
%
% NOTE: comment the +LP line and uncomment the -LP line to have page
%       numbers without the ``of ##'' last page reference)
%
% NOTE: uncomment the \pagestyle{empty} line to get rid of all page
%       numbers (make sure includefoot is commented out above)
%
\usepackage{fancyhdr,lastpage}
\pagestyle{fancy}
%\pagestyle{empty}      % Uncomment this to get rid of page numbers
\fancyhf{}\renewcommand{\headrulewidth}{0pt}
%\renewcommand{\headheight}{16pt}
\fancyhead[R]{\small{\textit{Petr Keil  (Curriculumn Vitae, page 
      \thepage~of~\protect\pageref*{LastPage}) }}}


% Finally, give us PDF bookmarks
\usepackage{color,hyperref}
\definecolor{darkblue}{rgb}{0.243,0.243,0.7}
\hypersetup{colorlinks,breaklinks,
            linkcolor=darkblue,urlcolor=darkblue,
            anchorcolor=darkblue,citecolor=darkblue}

%%%%%%%%%%%%%%%%%%%%%%%% End Document Setup %%%%%%%%%%%%%%%%%%%%%%%%%%%%


%%%%%%%%%%%%%%%%%%%%%%%%%%% Helper Commands %%%%%%%%%%%%%%%%%%%%%%%%%%%%

%%% HEADING AT TOP OF CURRICULUM VITAE

% The title (name) with a horizontal rule under it
% (optional argument typesets an object right-justified across from name
%  as well)
%
% Usage: \makeheading{name}
%        OR
%        \makeheading[right_object]{name}
%
% Place at top of document. It should be the first thing.
% If ``right_object'' is provided in the square-braced optional
% argument, it will be right justified on the same line as ``name'' at
% the top of the CV. For example:
%
%       \makeheading[\emph{Curriculum vitae}]{Your Name}
%
% will put an emphasized ``Curriculum vitae'' at the top of the document
% as a title. Likewise, a picture could be included:
%
%   \makeheading[\includegraphics[height=1.5in]{my_picutre}]{Your Name}
%
% the picture will be flush right across from the name.
\newcommand{\makeheading}[2][]%
        {\hspace*{-\marginparsep minus \marginparwidth}%
         \begin{minipage}[t]{\textwidth+\marginparwidth+\marginparsep}%
             {\large \bfseries #2 \hfill #1}\\[-0.15\baselineskip]%
                 \rule{\columnwidth}{1pt}%
         \end{minipage}}

%%% SECTION HEADINGS

% The section headings. Flush left in small caps down pseudo-margin.
%
% Usage: \section{section name}
\renewcommand{\section}[1]{\pagebreak[3]%
    \vspace{1.3\baselineskip}%
    \phantomsection\addcontentsline{toc}{section}{#1}%
    \noindent\llap{\scshape\smash{\parbox[t]{\marginparwidth}{\hyphenpenalty=10000\raggedright #1}}}%
    \vspace{-\baselineskip}\par}

%%% LISTS

% This macro alters a list by removing some of the space that follows the list
% (is used by lists below)
\newcommand*\fixendlist[1]{%
    \expandafter\let\csname preFixEndListend#1\expandafter\endcsname\csname end#1\endcsname
    \expandafter\def\csname end#1\endcsname{\csname preFixEndListend#1\endcsname\vspace{-0.6\baselineskip}}}

% These macros help ensure that items in outer-type lists do not get
% separated from the next line by a page break
% (they are used by lists below)
\let\originalItem\item
\newcommand*\fixouterlist[1]{%
    \expandafter\let\csname preFixOuterList#1\expandafter\endcsname\csname #1\endcsname
    \expandafter\def\csname #1\endcsname{\csname preFixOuterList#1\endcsname\let\oldItem\item\def\item{\pagebreak[2]\oldItem}}
    \expandafter\let\csname preFixOuterListend#1\expandafter\endcsname\csname end#1\endcsname
    \expandafter\def\csname end#1\endcsname{\let\item\oldItem\csname preFixOuterListend#1\endcsname}}
\newcommand*\fixinnerlist[1]{%
    \expandafter\let\csname preFixInnerList#1\expandafter\endcsname\csname #1\endcsname
    \expandafter\def\csname #1\endcsname{\let\oldItem\item\let\item\originalItem\csname preFixInnerList#1\endcsname}
    \expandafter\let\csname preFixInnerListend#1\expandafter\endcsname\csname end#1\endcsname
    \expandafter\def\csname end#1\endcsname{\csname preFixInnerListend#1\endcsname\let\item\oldItem}}

% An itemize-style list with lots of space between items
%
% Usage:
%   \begin{outerlist}
%       \item ...    % (or \item[] for no bullet)
%   \end{outerlist}
\newlist{outerlist}{itemize}{3}
    \setlist[outerlist]{label=\enskip\textbullet,leftmargin=*}
    \fixendlist{outerlist}
    \fixouterlist{outerlist}

% An environment IDENTICAL to outerlist that has better pre-list spacing
% when used as the first thing in a \section
%
% Usage:
%   \begin{lonelist}
%       \item ...    % (or \item[] for no bullet)
%   \end{lonelist}
\newlist{lonelist}{itemize}{3}
    \setlist[lonelist]{label=\enskip\textbullet,leftmargin=*,partopsep=0pt,topsep=0pt}
    \fixendlist{lonelist}
    \fixouterlist{lonelist}

% An itemize-style list with little space between items
%
% Usage:
%   \begin{innerlist}
%       \item ...    % (or \item[] for no bullet)
%   \end{innerlist}
\newlist{innerlist}{itemize}{3}
\renewcommand\labelitemi{}
\setlist[innerlist]{label=\textbullet,leftmargin=*,parsep=0pt,itemsep=0pt,topsep=0pt,partopsep=0pt}
\fixinnerlist{innerlist}

% An environment IDENTICAL to innerlist that has better pre-list spacing
% when used as the first thing in a \section
%
% Usage:
%   \begin{loneinnerlist}
%       \item ...    % (or \item[] for no bullet)
%   \end{loneinnerlist}
\newlist{loneinnerlist}{itemize}{3}
    \setlist[loneinnerlist]{label=\enskip\textbullet,leftmargin=*,parsep=0pt,itemsep=0pt,topsep=0pt,partopsep=0pt}
    \fixendlist{loneinnerlist}
    \fixinnerlist{loneinnerlist}

%%% EXTRA SPACE

% To add some paragraph space between lines.
% This also tells LaTeX to preferably break a page on one of these gaps
% if there is a needed pagebreak nearby.
\newcommand{\blankline}{\quad\pagebreak[3]}
\newcommand{\halfblankline}{\quad\vspace{-0.5\baselineskip}\pagebreak[3]}

%%% FORMATTING MACROS

% Uses hyperref to link DOI
\newcommand\doilink[1]{\href{http://dx.doi.org/#1}{#1}}
\newcommand\doi[1]{doi:\doilink{#1}}

% For \url{SOME_URL}, links SOME_URL to the url SOME_URL
\providecommand*\url[1]{\href{#1}{#1}}
% Same as above, but pretty-prints SOME_URL in teletype fixed-width font
\renewcommand*\url[1]{\href{#1}{\texttt{#1}}}

% For \email{ADDRESS}, links ADDRESS to the url mailto:ADDRESS
\providecommand*\email[1]{\href{mailto:#1}{#1}}
% Same as above, but pretty-prints ADDRESS in teletype fixed-width font
%\renewcommand*\email[1]{\href{mailto:#1}{\texttt{#1}}}

%\providecommand\BibTeX{{\rm B\kern-.05em{\sc i\kern-.025em b}\kern-.08em
%    T\kern-.1667em\lower.7ex\hbox{E}\kern-.125emX}}
%\providecommand\BibTeX{{\rm B\kern-.05em{\sc i\kern-.025em b}\kern-.08em
%    \TeX}}
\providecommand\BibTeX{{B\kern-.05em{\sc i\kern-.025em b}\kern-.08em
    \TeX}}
\providecommand\Matlab{\textsc{Matlab}}

% Custom hyphenation rules for words that LaTeX has trouble with
\hyphenation{bio-mim-ic-ry bio-in-spi-ra-tion re-us-a-ble pro-vid-er}

%%%%%%%%%%%%%%%%%%%%%%%% End Helper Commands %%%%%%%%%%%%%%%%%%%%%%%%%%%

\usepackage{wasysym}

%%%%%%%%%%%%%%%%%%%%%%%%% Begin CV Document %%%%%%%%%%%%%%%%%%%%%%%%%%%%

\begin{document}

\makeheading{Petr Keil -- Curriculum Vitae}

\thispagestyle{empty}

\section{Contact information}

	German Centre for Integrative Biodiversity Research -- iDiv \\
	Deutscher Platz 5e,  04103 Leipzig, Germany\\
    E-mail: \email{pkeil@seznam.cz}\\
    Personal web and blog: \href{http://www.petrkeil.com}	{www.petrkeil.com}\\

\HRule

\section{Education}

\begin{innerlist}

\item[]{\bf  Ph.D., Ecology} 
\hfill {2006--2010} \\
Department of Ecology, Faculty of Science, {\bf Charles University in Prague}, Czech Republic. Dissertation: \textit{Macroecology of European invertebrates: spatial and temporal patterns extracted from heterogeneous data}. Advisor: \href{http://www.cts.cuni.cz/~storch/}{David Storch}.\\

\item[]{\bf  M.S., Zoology \& Entomology} 
\hfill {2003--2005} \\
Department of Zoology, Faculty of Biology, {\bf University of South Bohemia}, Ceske Budejovice, Czech Republic. Advisor: \href{http://www.entu.cas.cz/en/staff/Martin-Konvicka-r81r/?all_publications=show}{Martin Konvicka}. I spent one semester of 2004 at {\bf Universit\'{e} de Rennes 1}, France, as an EU-funded Erasmus exchange student.\\ 

\item[]{\bf  B.S., {Biology}} 
\hfill {2000--2003} \\
Faculty of Biology, University of South Bohemia, Ceske Budejovice, Czech Republic\\

\end{innerlist}

\HRule

\section{Professional experience}

\begin{innerlist}

\item[]{\bf Postdoc} 
\hfill {2015--present} \\
German Centre for Integrative Biodiversity Research -- \textbf{iDiv}, Leipzig, Germany.\\

\item[]{\bf Postdoc} 
\hfill {2011--2014} \\
Dept. of Ecology and Evolutionary Biology, {\bf Yale University}, USA. One year funded by \href{http://www.mappinglife.org/}{Map of Life} project of \href{http://jetzlab.yale.edu/people/walter-jetz}{Walter Jetz}. Since 2012 funded by EU Marie-Curie fellowship.\\

\item[]{\bf Research associate} 
\hfill {2011, 2012--2015} \\ 
Center for Theoretical Study -- CTS, {\bf Charles University in Prague}, Czech Republic. \\

\item[]{\bf Early--stage research trainee} 
\hfill {2008 (5 months), 2009 (2 months)} \\ 
Faculty of Biological Sciences, {\bf University of Leeds}, UK. 
Funded by a Marie Curie Fellowship for early--stage researchers. Advisors: \href{http://www.fbs.leeds.ac.uk/staff/profile.php?tag=Biesmeijer}{Koos Biesmeijer} \& \href{http://www.fbs.leeds.ac.uk/staff/profile.php?tag=Kunin}{William E. Kunin}.\\

\item[]{\bf Research associate} 
\hfill {2007--2010} \\ 
Dept. of Ecology, Faculty of Science, Charles University in Prague, Czech Republic. The position was funded from various grants, and was held parallel to my Ph.D. studies.\\

\item[]{\bf Technician}
\hfill {2006}\\
Department of Zoology, {\bf University of Cambridge}, UK. Six months as a technician in the lab of \href{http://www.zoo.cam.ac.uk/zoostaff/hedwig/index.html}{Berthold Hedwig}. Performed behavioural experiments with crickets.\\\\\\\\

\end{innerlist}

\newpage

\HRule

\section{Peer-reviewed articles}

\begin{etaremune}

\item Keil. P, Pereira H.M., Cabral J.S., Chase J.M., May F., Martins I.S. \& Winter M. (2017) Spatial scaling of extinction rates: theory and data reveal non-linearity, and a major upscaling and downscaling challenge. \textit{\textbf{Global Ecology and Biogeography}}, in press.


\item Gratton P., Marta S., Bocksberger G., Winter M., Keil P., Trucchi E. \& Kühl H. (2017) Which latitudinal gradients for genetic diversity? \textit{\textbf{Trends in Ecology and Evolution}}, in press.

\item Keil P., Storch D. \& Jetz W. (2015) On the decline of biodiversity due to area loss. \textit{\textbf{Nature Communications}}, 6: 8837.

\item Quintero I., Keil P., Jetz W. \& Crawford F.W. (2015) Historical biogeography using species geographical ranges. \textit{\textbf{Systematic Biology}}, 64: 1059-1073. [\href{http://sysbio.oxfordjournals.org/content/64/6/1059.short}{link}] 

\item Tropek R., Sedlacek O., Beck J., Keil P., Musilova Z., Simova I. \& Storch D. (2014) Comment on ``High resolution global maps of 21-st century forest cover change". \textit{\textbf{Science}}, 344: 981. [\href{http://www.sciencemag.org/content/344/6187/981.4.full}{link}] 

\item Keil P., Wilson A.O. \& Jetz W. (2014) Uncertainty, priors, autocorrelation and disparate data in downscaling of species distributions. \textit{\textbf{Diversity and Distributions}}, 20: 797-812. [\href{http://onlinelibrary.wiley.com/doi/10.1111/ddi.12199/abstract}{link}]

\item Sturrock H.J.W, Cohen J.M., Keil P., Tatem A.J., Menach A.L., Ntshalinthsali N.E., Hsiang M.S. \& Gosling R.D. (2014) Fine-scale malaria risk mapping from routine aggregated data. \textit{\textbf{Malaria Journal}}, 13: 421.

\item Keil P. \& Jetz W. (2014) Downscaling the environmental associations and spatial patterns of species richness. \textit{\textbf{Ecological Applications}}, 20: 797-812. [\href{http://www.esajournals.org/doi/abs/10.1890/13-0805.1?af=R&}{link}] 

\item Pechacek P., Stella D., Keil P. \& Kleisner K. (2014) Environmental effects on the shape variation of male ultraviolet patterns in the brimstone butterfly (\textit{Gonepteryx rhamni}, Pieridae, Lepidoptera). \textit{\textbf{Naturwissenschaften}}, 101, 1055-1063.  [\href{http://link.springer.com/article/10.1007/s00114-014-1244-5}{link}] 

\item Straka J., Cerna K., Machackova L., Zemenova M. \& Keil P. (2014) Life span in the wild: the role of activity and climate in natural populations of bees. \textit{\textbf{Functional Ecology}}, 28: 1235-1244.
[\href{http://onlinelibrary.wiley.com/doi/10.1111/1365-2435.12261/abstract}{link}]

\item Keil P., Belmaker J., Wilson A.M., Unitt P. \& Jetz W. (2013) Downscaling of species distribution models: a hierarchical approach. \textit{\textbf{Methods in Ecology and Evolution}}, 4: 82-94. [\href{http://onlinelibrary.wiley.com/doi/10.1111/j.2041-210x.2012.00264.x/abstract}{link}]

\item Carvalheiro L., Kunin W.E., Keil P. \& 16 authors (2013) Biodiversity declines and biotic homogenization have slowed down for NW--European pollinators and plants. \textit{\textbf{Ecology Letters}}, 16: 870-878. [\href{http://onlinelibrary.wiley.com/doi/10.1111/ele.12121/abstract}{link}]

\item Storch D., Keil P. \& Jetz W. (2012) Universal species--area and endemics--area relationships at continental scales. \textit{\textbf{Nature}}, 488: 78-81. [\href{http://www.nature.com/nature/journal/v488/n7409/full/nature11226.html}{link}]

\item Keil P., Schweiger O., Kuhn I., Kuussaari M., Kunin W.E., Settele J., Henle K., Brotons L., Pe’er G., Lengyel S., Moustakas A., Steinicke H. \& Storch D. (2012) Patterns of beta diversity in Europe: the role of climate, land cover and distance across scales. \textit{\textbf{Journal of Biogeography}}, 39: 1473-1486. [\href{http://onlinelibrary.wiley.com/doi/10.1111/j.1365-2699.2012.02701.x/abstract}{link}]

\item Kleisner K., Keil P. \& Jaros F. (2012) Biogeography of elytral ornaments in Palearctic genus Carabus: disentangling the effects of space, evolution and environment at a continental scale. \textit{\textbf{Evolutionary Ecology}}, 26: 1025-1040. [\href{http://link.springer.com/article/10.1007%2Fs10682-011-9537-z}{link}]

\item Henle et al. (2012) Nature Conservation -- a new dimension in Open Access publishing bridging science and application. \textit{\textbf{Nature Conservation}}, 1: 1-10. [\href{http://www.pensoft.net/journals/natureconservation/article/3081/abstract/}{link}]

\item Keil P., Biesmeijer J.C., Barendregt A., Reemer M. \& Kunin W.E. (2011) Biodiversity change is scale--dependent: an example from dutch and UK hoverflies (Diptera, Syrphidae). \textit{\textbf{Ecography}}, 34: 392-401. [\href{http://onlinelibrary.wiley.com/doi/10.1111/j.1600-0587.2010.06554.x/abstract}{link}]

\item Simova I., Storch D., Keil P., Boyle B., Phillips O.L. \& Enquist B.J. (2011) Global species--energy relationship in forest plots: role of abundance, temperature and species' climatic tolerances. \textit{\textbf{Global Ecology and Biogeography}}, 6: 842-856. [\href{http://onlinelibrary.wiley.com/doi/10.1111/j.1466-8238.2011.00650.x/abstract}{link}] 

\item Keil P., Herben T., Rosindell J. \& Storch D. (2010) Predictions of Taylor's power law, density dependence and pink noise from a neutrally modelled time series. \textit{\textbf{Journal of Theoretical Biology}}, 265: 78-86. [\href{http://www.sciencedirect.com/science/article/pii/S0022519310001979}{link}]

\item Sizling A.L., Storch D. \& Keil P. (2009) Rapoport's rule, species tolerances, and the latitudinal diversity gradient: geometric considerations. \textit{\textbf{Ecology}}, 90: 3575-3586. [\href{http://www.esajournals.org/doi/abs/10.1890/08-1129.1?journalCode=ecol}{link}]

\item Keil P. \& Hawkins B.A. (2009) Grids versus regional species lists: are broad--scale patterns of species richness robust to the violation of constant grain size? \textit{\textbf{Biodiversity and Conservation}}, 18: 3127-3137. [\href{http://link.springer.com/article/10.1007%2Fs10531-009-9631-5}{link}]

\item Dixon A.F.G., Honek A., Keil P., Kotela M.A., Sizling A.L. \& Jarosik V. (2009) Relationship between the minimum and maximum temperature thresholds for development in insects. \textit{\textbf{Functional Ecology}}, 23: 257-264. [\href{http://onlinelibrary.wiley.com/doi/10.1111/j.1365-2435.2008.01489.x/abstract}{link}]

\item Keil P., Dziock F. \& Storch D. (2008) Geographical patterns of hoverfly (Diptera, Syrphidae) functional groups in Europe: inconsistency in environmental correlates and latitudinal trends. \textit{\textbf{Ecological Entomology}}, 33: 748-757. [\href{http://onlinelibrary.wiley.com/doi/10.1111/j.1365-2311.2008.01032.x/abstract}{link}]

\item Langrova I., Makovcova K., Vadlejch J., Jankovska I., Petrtyl M., Fetchner J., Keil P., Lytvynets A. \& Borkovcova M. (2008) Arrested development of sheep strongyles: onset and resumption under field conditions of Central Europe. \textit{\textbf{Parasitology Research}}, 103: 387-392. [\href{http://link.springer.com/article/10.1007%2Fs00436-008-0984-6}{link}]

\item Keil P., Simova I. \& Hawkins, B.A. (2008) Water-energy and the geographical species richness pattern of European and North African dragonflies (Odonata). \textit{\textbf{Insect Conservation and Diversity}}, 1: 142-150. [\href{http://onlinelibrary.wiley.com/doi/10.1111/j.1752-4598.2008.00019.x/abstract}{link}]

\item Konvicka M., Novak J., Benes J., Fric Z., Bradley J., Keil P., Hrcek J., Chobot K. \& Marhoul P. (2007) The last population of the Woodland Brown butterfly (Lopinga achine) in the Czech Republic: habitat use, demography and site management. \textit{\textbf{Journal of Insect Conservation}}, 12: 549-560. [\href{http://link.springer.com/article/10.1007%2Fs10841-007-9087-4}{link}]

\item Keil P. \& Konvicka M. (2005) Local species richness of Central European hoverflies (Diptera: Syrphidae): a lesson taught by local faunal lists. \textit{\textbf{Diversity and Distributions}}, 11: 417-426. [\href{http://onlinelibrary.wiley.com/doi/10.1111/j.1366-9516.2005.00172.x/abstract}{link}]

\item Keil P. (2005) Microhabitat preferences of Stiletto-flies larvae (Diptera: Therevidae) in mountain primeval forest. \textit{\textbf{Studia Dipterologica}}, 12(1): 87-92. [\href{http://www.studia-dipt.de/con121.htm}{link}]
\end{etaremune}

\HRule

\section{Chapters, preprints, proceedings}
\begin{etaremune} 

\item Keil P., Cabral J.S., Chase J., Martins I.S., May F., Pereira H.M. \& Winter M. (2016) Extinction rate has a complex and non-linear relationship with area. \textit{\textbf{BioRxiv}}, 4:e2367v2. [\href{http://biorxiv.org/content/early/2016/10/18/081489}{link}]

\item Keil P., Bennett J.M., Burgeois B., Garcia-Pena G.E., MacDonald A.M., Meyer C., Ramirez K.S. \& Yguel K.S. (2016) From computer operating systems to biodiversity: co-emergence of ecological and evolutionary patterns. \textit{\textbf{PeerJ Preprints}}, 4:e2367v2. [\href{https://peerj.com/preprints/2367/}{link}]

\item Bitner J., Schejbalova A., Bartova K., Varella J., Krejcova L., Weiss P., Kleisner K., Keil P. \& Kvapilova K. (2015) Prevalence of zoophilia in Czech sado-masochistic community. \textit{\textbf{Journal of Sexual Medicine}}, 12: 41-41.

\item Keil P. (2014) Limits of uncertainty about estimates of probability of ecological events. \textit{\textbf{PeerJ PrePrints }}, 2:e446v1. [\href{https://peerj.com/preprints/446/}{link}]

\item Storch D., Keil P. \& Kunin W.E. (2014) Scaling communities and biodiversity. Pages 66-77. In: \textit{\textbf{Scaling in Ecology and Biodiversity Conservation}}, Pensoft. Eds: Henle et al. [\href{http://ab.pensoft.net/articles.php?id=1169}{link}]
 
\end{etaremune}

\HRule

\section{Service for Journals}

Associate editor of \textit{\textbf{Global Ecology and Biogeography}} (Wiley)

\medskip

Associate editor of \textit{\textbf{Nature Conservation}} (Pensoft Publishers) \\
\medskip

{\bf Reviewer for}: \textit{Ecology Letters, Nature Ecology \& Evolution, Ecography, Journal of Biogeography, Global Ecology and Biogeography, Journal of Animal Ecology, Diversity and Distributions, Oikos, Perspectives in Plant Ecology Evolution and Systematics, Journal of Geographic Information Systems, Acta Oecologica, Methods in Ecology and Evolution, Biodiversity and Conservation, Biological Conservation, Population Ecology, Ecological Modelling, European Journal of Entomology, Polish Journal of Ecology, Journal of Vegetation Science, Ecology \& Evolution, Ethology Ecology and Evolution, Ecological Applications, American Naturalist, Zapadoceske Entomologicke Listy.}

\HRule

%\newpage
\section{Citation \\ Metrics}

Accessed 23 Aug 2017:

\begin{itemize}
\item \textbf{Google Scholar:} 822 citations, h-index  = 15.
\item \textbf{ISI Web of Science:} 546 citations (529 without self-citations), h-index = 13.
\end{itemize}


\HRule

\section{Students}

\begin{innerlist}

\item[]{\bf Maria Sporbert}, PhD student at Martin Luther University of Halle-Wittenberg (co-advisor, ongoing)


\medskip

\item[]{\bf Adam Klimes}, Master student at Charles University (advisor, finished in 2016)

\end{innerlist}

\HRule

\section{Lecturing}

\begin{innerlist}


\item[]{\bf Bayesian Biostatistics} - University of Leipzig, Charles University
\hfill {2014-2017}\\
Three-day intensive practical course for 15-20 students.
[\href{http://www.petrkeil.com/?p=1881}{link}]

\medskip

\item[]{\bf Spatial Biodiversity Analysis Methods} - Yale University, USA
\hfill {2013}\\
Taught together with Adam M. Wilson, Giuseppe Amatulli and Walter Jetz.
[\href{https://sites.google.com/site/spatialbiodiversity/home}{link}]

\medskip

\item[]{\bf Spatial Scale in Biodiversity Science} - Yale University, USA
\hfill {2012} \\
Graduate course (Yale EEB 713 01 - S13) for ca. 10 students, taught with prof. Walter Jetz. [\href{http://pantheon.yale.edu/~pk327/pdf/Keil&Jetz_syllabus.pdf}{link to syllabus}]

\medskip

\item[]{\bf Course of R Programming} - Charles University, Czech Republic
\hfill {2009--2011} \\
Graduate course of R programming \textit{``R for Life"} for ca. 20 students at Faculty of Science.

\medskip

\item[]{\bf Foundations of Ecology} - Charles University, Czech Republic
\hfill {2007--2009} \\
Principal lecturer of graduate course \textit{``Foundations of Ecology"} for ca. 20 students of Social and Cultural Ecology at Faculty of Humanities (FHS).

\end{innerlist}

\HRule


\section{3rd party funding}

\begin{innerlist}

\item[]{\bf sDiv Fellowship, iDiv, Leipzig, Germany}
\hfill {2015--2017} \\
2-year independent postdoctoral project ``\textit{Assessing extinction: Crossover of geometrical, macroecological and Bayesian perspectives}".
\\

\item[]{\bf Marie Curie Fellowship (IOF)}
\hfill {2012--2015} \\
EU FP7-PEOPLE, Project WORLDIVERSITY -- ``\textit{Linking global species and beta diversity to individual species distributions at multiple phylogenetic and spatial scales}"  (\# 302868). Host: Yale University, USA.
\\

\item[]{\bf Grand Agency of Charles University}
\hfill {2008--2010} \\
Co-PI (with Irena Simova) on project GAUK106108  \textit{``The effect of energy availability and environmental heterogeneity on species diversity of plants and invertebrates"}. 

\end{innerlist}

\HRule

\section{Awards}

Dean's prize for the \textbf{best PhD thesis} defended in 2010 at Faculty of Science, Charles University in Prague.

\HRule

\section{Skills}

{\bf Software}: R, C++, Windows, Linux, Bash, Grass GIS, ArcGIS, QGIS, OpenBUGS, JAGS, STAN, Canoco, \LaTeX, HTML, CSS, SQL, Markdown, WordPress, Graphic editors (GIMP, InkScape, Corel and Adobe suites).

\medskip

{\bf Statistics}: GLM, hierarchical (mixed-effects) models, GAM, ordinations, spatial statistics, time series analysis, survival analysis, structural equations models, data visualization, machine learning, model selection, model diagnostics, predictive inference, Monte-Carlo tests, bootstrap, Bayesian inference, likelihood optimization.

\medskip

{\bf Languages}: English (excellent); German, Spanish, French (intermediate); Czech (nat.).

\HRule

\section{References}

Dr. \textbf{Marten Winter}, iDiv. \phone +49 3419733129. \href{mailto: marten.winter@idiv.de}{marten.winter@idiv.de}

\medskip

Prof. \textbf{Walter Jetz}, Yale University. \phone +1 2034327540. \href{mailto:walter.jetz@yale.edu}{walter.jetz@yale.edu}

\medskip

Prof. \textbf{David Storch}, Charles University. \phone +420 221183535. 
\href{mailto:storch@cts.cuni.cz}{storch@cts.cuni.cz}

\medskip

Prof. \textbf{William Kunin}, University of Leeds. \phone +440 1133432857. \href{mailto:w.e.kunin@leeds.ac.uk}{w.e.kunin@leeds.ac.uk}






\end{document}\grid
\grid
